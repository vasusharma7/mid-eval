%_____________________________________________________________________________________________ 
% LATEX Template: Department of Comp/IT BTech Project Reports
% Sample Chapter
% Sun Mar 27 10:25:35 IST 2011
%
% Note: Itemization, enumeration and other things not shown. A sample figure is included.
%_____________________________________________________________________________________________ 

\setstretch{1.5}
{\let\clearpage\relax \chapter{Conclusion}}

We started out with the goal of getting better at text recognition, focusing on handwritten text and identified 3 paths - text segmentation, improving image quality and doing character level segmentation and recognition. We started on the path of text segmentation and tried 2 network archtiectures - ARU Net and CRAFT. As discussed in the results section, ARU Net just gives the baselines and we heavily depend on our post processing techniques to reach the text segmentation we want. This approach does a descent job in identifying the lines but there are certain cases, still not addressed,  regarding mis predictions in case of complex handwritten data and irregular number of boxes for a line in case of machine print images. For CRAFT, using supervised learning, we weren't able to reach good results since its difficult to gather character level bounding box data for handwritten documents and the dataset we used, also had several inaccurate character level bounding boxes. We got better at recognising the handwritten text while continuing doing good on machine print document,but the model is not enough good as of now to be production ready.

\textbf{Ongoing Experiments and future scope:} We are exploring the path of semi-supervised learning to get better at segmentation using CRAFT where in we provide the word level bounding boxes and let the the network figure out the character level regions using the interim model trained on synthetic data. Several experiments are also being done for training the model in shorter time, using multiple dataloaders, parallel processing and maximising the GPU usage. We are also planning to work on improving the handwritten OCR model so as to get better at recognising handwriting after segmentation. Some optimisations in the post processing techniques are also underway to get better text recognition results. We have also planned certain changes to original CRAFT implementation in terms of changing the network architecture to do transfer learning to get to the kind of model outputs that help us more in post processing. All such experiments are being worked upon. 

%_____________________________________________________________________________________________ 
