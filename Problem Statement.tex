% TODO: \usepackage{graphicx} required
% \begin{figure}
% 	\centering
% 	\includegraphics[width=0.7\linewidth]{"system diagram"}
% 	\caption{Text Segmentation and OCR: System diagram}
% 	\label{fig:system-diagram}
% \end{figure}

%_____________________________________________________________________________________________ 
% LATEX Template: Department of Comp/IT BTech Project Reports
% Sample Chapter
% Sun Mar 27 10:25:35 IST 2011
%
% Note: Itemization, enumeration and other things not shown. A sample figure is included.
%_____________________________________________________________________________________________ 

\setstretch{1.5}
{\let\clearpage\relax \chapter{Problem Statement and Research Objectives}}

\textbf{To improve Optical Character Recognition for handwritten and machine print documents by doing Text detection, localisation and segmentation}
\newline

Although the established techniques work well for segmentation of straight-forward machine print, images distorted due to camera angles, brightness, focus etc, containing handwritten content, containing mixed text orientation and fonts etc pose a significant challenge for text localization as well as text recognition. Errors in the earlier segmentation stage compund the problems in subsequent stages of the pipeline resulting in poor outcomes.

This problem can be addressed by using deep learning based models for text localisation, where in we identify specific areas where the text is located in the page, extract those regions and pass them to the line or word-level recognition models. The deep learning models are so developed that they are agnostic to camera angles, noise and many other distortions in the image.

As an additional improvement, we also want to get character level boundaries in the image along with word or line regions/boundaries. This is a crucial feature in reducing the human effort needed in cleaning up OCR or information extraction results, as the manual effort can now be directed to verify particular characters in the image with lower confidence.


\vspace{10pt}
\vspace{10pt}
{\Large{\textbf{Objectives}}}
\begin{enumerate}
	
	\item To develop deep learning based models for text detection, localisation and segmentation in the docuements of handwritten and  machine print data, eventually to get better at recognizing the words and lines in the documents using already developed models.
	
	\item To improve word and line OCR models for text recognition post text detection step, speciifically for handwritten data.
	
	\item To get better at recognising text in the bad quality images with distortions like noise, random camera angles, brightness, different ink colour etc.
	
	\item To be able to precisely identify character level boundaries along with recognising words in the image so as to be able to re-direct the human user to verify the characters with lower confidence.
	
\end{enumerate}
%_____________________________________________________________________________________________ 
